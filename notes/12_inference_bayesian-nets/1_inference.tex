\section{Inference and Uncertainty}

\begin{frame} 

\begin{center} \footnotesize
\textit{``As far as the laws of mathematics refer to reality, they are not certain; and as far as they are certain, they do not refer to reality."}
\end{center} 
\begin{flushright} \footnotesize
Albert Einstein, 1921
\end{flushright}

\mode<presentation>{
    \vspace{10mm}
    \begin{center} \huge
        \secname
    \end{center}
    \vspace{10mm}
    }
Because we need to
    \begin{itemize}
    \item[] quantify our degree of belief/\\
	\item[] quantify our uncertainty
    \end{itemize}
    
\end{frame}

\subsection{The probabilistic model}

\begin{frame}\frametitle{\subsecname}

\begin{itemize}
    \item \underline{The model:}\\
        \begin{align}
            P(H): H \rightarrow [0,1]&\\
            P(H) = 0 & \quad\iff H \text{ is false (i.e. impossible)} \\
            P(H) = 1 & \quad\iff H \text{ is true (i.e. certain)} \\
            0 < P(H) < 1 & \quad\;\;\text{quantifies our degree of belief}
        \end{align}
    \pause
    \item \underline{events}: assignment to a set of cases\\
    (e.g. two dice add up to 11)
    \pause
    \item \underline{proposition}: disjunction of \textbf{events}\\
    (i.e. isolate the sets of \textbf{events} in which the \textbf{proposition} holds)\\
    
\end{itemize}
    
\end{frame}

\begin{frame}

$\underbrace{\text{\emph{Summing over}}}_{\text{marginalisation}}$ the probabilities of the \textbf{events} $\leadsto$ probability of \textbf{proposition}.\\

Example:

\begin{align}
P(\text{Total} = 11) &= P(\text{Die}_{1} = 5, \text{Die}_{2} = 6) + P(\text{Die}_{1} = 6, \text{Die}_{2} = 5)\\
        &\stackrel{\text{shorthand}}{=} P(5,6) + P(6,5)\\
        &= 1 / 36 + 1/36\\
        &= 1/18
\end{align}

$P(\text{Total} = 11), P(5,6) \text{ and } P(6,5)$ are \emph{unconditional} or \emph{prior} probabilities.\\
They represent the degree of belief in the absence of any other information.\\
Priors is also referred to as \emph{domain knowledge}.
        
\end{frame}

\subsection{Evidence}

\begin{frame}\frametitle{\subsecname}

What if we already \emph{know} (i.e. observed) something?\\

e.g. We've already rolled the first die and got 5.\\
Now we're waiting to roll the second die.\\

We are now interested in the \emph{conditional} (or \emph{posterior}) probability of the second die leading to a total of 11:

\begin{equation}
P(\text{Total}= 11) \underbrace{\big|}_{\text{``given''}} \overbrace{\text{Die}_{1}}^{\mathclap{\{1,2,3,4,5,6\}}} = 5)
\end{equation}

\end{frame}

\begin{frame}\frametitle{A thing about priors}

The prior is valid event after the outcome.

e.g.\\
\begin{itemize}
 \item[] $P(\overbrace{cavity}^{\text{Cavity}=1}) = 0.2$
 \item[] $P(cavity | \overbrace{toothache}^{\text{Toothache}=1}) = 0.6$
 \item[] If we observe that a cavity has actually occured, we don't have to change our prior on cavities.\\
 However, the prior becomes less useful if we proceed to infer other things.
\end{itemize}
    
\textbf{Caution}\\
\begin{itemize}
\item[$\times$] Whenever Toothache is true, then cavity is true with probability of 0.6
\item[\checkmark] Whenever Toothache is true, \textit{and no other information} is available, then cavity is true with probability of 0.6.
\end{itemize}

\textit{other information}: Diagnosis was \underline{no} cavity. In this case, the probability is no longer 0.6, but:
\begin{equation}
P(cavity | toothache \wedge Diagnosis cavity = false) = 0
\end{equation}

\end{frame}

Observing event $e$ rules out events where e is false. This leaves a set where $P(e)>0$.
We look within that set for the fraction that satisfies $x$ and $e$.

This fraction can be calculated using the following:

\begin{equation}
P(x|e) = \frac{P(x,e)}{P(e)}
\end{equation}

which brings us to the \emph{product rule}:

\begin{equation}
P(x,e) = P(x|e) P(e)
\label{eq:productrule}    
\end{equation}


\begin{frame}

\question{What information do I need to calculate the probability of any proposition?}

- The full joint probability distribution\notesonly{ provides all the information to calculate the probability of any proposition.}

Example:\\

%\footnotesize
	%\begin{center}
     \begin{table}[h] 
	\begin{tabular}{|c|c|c|}
		\hline
		& $toothache = \text{true}$ & $toothache = \text{false}$ \\
		\begin{tabular}{c}
			\\ 
			$cavity = \text{true}$ \\
			$cavity = \text{false}$
		\end{tabular} 
		& 
		\begin{tabular}{c|c}
			$catch = \text{true}$ & $catch = \text{false}$\\ 
			\hline 
			$0.108$ & $0.012$ \\
			$0.016$ & $0.064$
		\end{tabular} 
		& 
		\begin{tabular}{c|c}
			$catch = \text{true}$ & $catch = \text{false}$\\ 
			\hline 
			$0.072$ & $0.008$ \\
			$0.144$ & $0.576$
		\end{tabular} 
		\\ \hline
	\end{tabular}
    \end{table}
	

\end{frame}

\subsubsection{Marginalisation}



\begin{frame}\frametitle{\subsubsecname~(``summing out'')}

\notesonly{\subsecname~$\corresponds$~``summing out''}

\begin{align}
P(x | \vec e)
&= \frac{x, \vec e}{P(\vec e)}\\
&= \alpha \sum_{\vec y \in \vec Y} P(\vec x, \vec e, \vec y)
\end{align}

where
\begin{itemize}
\item[] $x$: query variable
\item[] $\vec e$: evidence variable(s)
\item[] $\vec y$: \textbf{un}observed variables in the table
\item[] $\alpha$: normalization factor:

\question{How do we calculate the normalization factor $\alpha$?}

- Via
\begin{enumerate}
\item evidence prior:
\begin{equation}
\alpha := P(\vec e) = \sum_{x, \vec y} P(x, \vec e, \vec y)  
\end{equation}
\item or simply via whatever ensures $\sum_{\vec x} P(x, \vec e) = 1$
\end{enumerate}

\end{itemize}
    
\end{frame}

\subsubsection{Applying the product rule}

\begin{frame}\frametitle{\subsubsecname}
    
\begin{align}
P(cavity | toothache=true) &= \frac{cavity,toothache}{P(toothache)}\\
&= \frac{0.108 + 0.012}{0.108 + 0.012 + 0.016 + 0.064}\\
&= 0.6   
\end{align}


\begin{equation*}
P(cavity=false | toothache=true) = ?
\end{equation*}

\question{What should we expect?}

\pause

\begin{align}
P(cavity=false | toothache=true) &= \frac{cavity=false,toothache}{P(toothache)}\\
&= \frac{0.016 + 0.0064}{0.108 + 0.012 + 0.016 + 0.064}\\
&= 0.4 = 1 - 0.6   
\end{align}

\textbf{Remark}: The denominator remains constant no matter which value of cavity we are interested in.\\
$\Rightarrow$ use $1/P(toothache)$ as the normalization factor for $P(cavity | toothache)$.

\begin{align}
P(Cavity|toothache) &= \alpha P(Cavity, toothache) \\
&= \alpha \big\lbrack
P(Cavity, toothache, catch) + P(Cavity, toothache, catch=false) 
\big\rbrack\\
&= \alpha \big\lbrack{}
\rmat{0.108\\0.016} + \rmat{0.012\\0.064}
\big\rbrack\\
&= \alpha \rmat{0.12\\0.08} \stackrel{\alpha=1/P(toothache)}{=} \rmat{0.6\\0.4}
\intertext{OR normalize s.t. $\sum_{\vec x} P(x, \vec e) \eqexcl 1$}
&= \alpha \rmat{0.12\\0.08} \stackrel{\alpha=\frac{1}{0.12+0.08}}{=} \rmat{0.6\\0.4}
\end{align}



\end{frame}
