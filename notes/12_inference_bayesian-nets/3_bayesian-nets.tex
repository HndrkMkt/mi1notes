\section{Bayesian Networks}

see Russel Norvig Ch. 14

\mode<presentation>{
\begin{frame} 
    \begin{center} \huge
        \secname
    \end{center}
    \begin{center}   
    A representation of the full joint distribution that exploits conditional independence
    \end{center}
\end{frame}
}

\begin{frame}\frametitle{\secname}
    
A factorization of the full joint distribution:

\begin{equation}
P(X_{1},\ldots,X_{n}) = \prod_{i=1}^{n} P(X_{i} | parents(X_{i}))
\end{equation}
    
Remember: independence is important. Independence can be identified via the 
\underline{Markov blanket}

The Markov blanket of a node includes:
\begin{itemize}
\item its parents
\item its children
other parents of the children    
\end{itemize}
    
\end{frame}

\subsection{Preparations to perform inference}

\begin{frame}\frametitle{\subsecname}
    
    \begin{enumerate}
     \item use topological representation to ensure a compact representation
     \item turn into a moral graph i.e.
     \begin{enumerate}
     \item turn DAG into an undirected graph + new edges s.t. each node of the original DAG is now directly connected to the nodes of its \emph{Markov blanket}
     \item to be continued next week.
     \end{enumerate}   
    \end{enumerate}
    
\end{frame}

\begin{frame}

Example DAG to moral graph

\textbf{see blackboard}
    
\end{frame}
