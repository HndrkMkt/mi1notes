\section{Message Passing}

\definecolor{darkgreen}{rgb}{0,0.6,0}

\mode<presentation>{
\begin{frame} 
    \begin{center} \huge
        \secname
    \end{center}
    kinda like look-up-table
\end{frame}
}

\begin{frame}

Inference in  a junction tree:

\begin{enumerate}
\item clique potentials (non-negative function) for each clique (get definitions from DAG)
\item order initialization according to topological sorting (parent -> child)
\item add clique potential(s) for observed evidence(s)
\item perform message passing
\item extract marginals form messages
\item perform normalization
\end{enumerate}
\end{frame}

\subsection{Overview}

\begin{frame}\frametitle{Message Passing: Overview}

\begin{itemize}
\item essentially storage of local computations for faster look-up
\item will be used to calculate conditional probabilities in a junction tree by only using local ``simpler'' computations.
\item scales well with no. of veriables.
\end{itemize}

\end{frame}

\subsection{Procedure}


\begin{frame}\frametitle{\secname: \subsecname}

\begin{enumerate}
\item request pass: selecting the root (for tree structures any node can be the root, so not sensitive to choice of root node)
\item collect pass: collect messages from other neighbors
\item distribute pass: 
\end{enumerate}

\end{frame}
