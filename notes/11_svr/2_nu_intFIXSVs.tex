
\subsection{An intuition for $\nu$}\label{sec:intuitionnu}

\mode<presentation>{
\begin{frame}\frametitle{\subsecname}
\mode<presentation>{

     \placeimage{11}{3.5}{img/regression_1d_linear_margin_phi_cropped_small_err}{width=3.5cm}
}

\begin{minipage}{7.5cm}

\begin{block}{The primal problem for $\nu$-SVR (the parts where $\nu$ and $\varepsilon$ interact)}
\slidesonly{\vspace{-5mm}}
	\begingroup
	\footnotesize
     \begin{equation}
        \begin{array}{ll}
        \min_{\vec w, b,\{\varphi_\alpha\}, \{\varphi_\alpha^*\}, \varepsilon} & \\[-1mm]
        & \kern-11ex\frac{1}{2} \lVert \vec w \rVert_{2}^{2} + C \big\lbrack {\color{blue}\nu \varepsilon} + \frac{1}{p} {\color{magenta}\sum_{\alpha}^p \overbrace{(\varphi_\alpha + \varphi_\alpha^*)}^{\text{errors}}} \big\rbrack\\[2mm]
        \text{subject to} &
        \varepsilon~\ge~0\\
        \end{array}
        \label{eq:primalnuepsonly}
     \end{equation}
     
     \slidesonly{\vspace{-2mm}}
        with $C>0$, $\nu \ge 0$, $\alpha=1,\ldots,p$
     \endgroup
\end{block}
\end{minipage}

\slidesonly{\vspace{10mm}}

We can try to understand the role of $\nu$ by looking at cases for $\varepsilon$:
\begin{enumerate}
\item When $\varepsilon=0$ 
\item As $\varepsilon$ \textbf{increases} from $0$, thus lowering the \textcolor{magenta}{sum of errors}
\item As $\varepsilon$ \textbf{decreases} from some large positive value, thus increasing the fraction of support vectors $\frac{\#\mathrm{SV}}{p}$
\end{enumerate}
    
\end{frame}
}

\begin{frame}%{Only}\frametitle{\subsubsecname}

\mode<presentation>{

     \placeimage{11}{1}{img/regression_1d_linear_margin_phi_cropped_small_err}{width=3.5cm}
}

\notesonly{
We look at the part of the primal problem in which $\nu$ and $\varepsilon$ interact\footnote{This is based on my understanding of Ch. 9.3 from \citep{scholkopf2001learning} and the paper \citep{scholkopf2000new}}:
}

%\slidesonly{\vspace{-20mm}}

\begin{minipage}{7.5cm}

\begin{block}{%The primal problem for $\nu$-SVR (the parts where $\nu$ and $\varepsilon$ interact)
}
\slidesonly{\vspace{-5mm}}
	\begingroup
	\footnotesize
     \begin{equation}
        \begin{array}{ll}
        \min_{\vec w, b,\{\varphi_\alpha\}, \{\varphi_\alpha^*\}, \varepsilon} & \\[-1mm]
        & \kern-11ex\frac{1}{2} \lVert \vec w \rVert_{2}^{2} + C \big\lbrack {\color{blue}\nu \varepsilon} + \frac{1}{p} {\color{magenta}\sum_{\alpha}^p \overbrace{(\varphi_\alpha + \varphi_\alpha^*)}^{\text{errors}}} \big\rbrack\\[2mm]
        \text{subject to} &
        \varepsilon~\ge~0\\
        \end{array}
        \label{eq:primalnuepsonly}
     \end{equation}
     
     \slidesonly{\vspace{-2mm}}
        with $C>0$, $\nu \ge 0$, $\alpha=1,\ldots,p$
     \endgroup
\end{block}
\end{minipage}

\begin{enumerate}
\item<only@1> $\varepsilon = 0$:
\begin{itemize}
\item[$\Rightarrow$] All points are \textbf{outside} the $\varepsilon$-tube.{}
\item[$\Rightarrow$] All points are SVs, i.e. $\frac{\#\mathrm{SV}}{p}=1$.
\item[$\Rightarrow$] The \textcolor{magenta}{sum of errors} will be $> 0$
\end{itemize}
\pause
\item<only@2-6>As $\varepsilon$ \textbf{increases} from $0$:
\begin{itemize}
\item[$\Rightarrow$]<only@3> \notesonly{The term }${\color{blue}\nu\varepsilon}$ \textbf{increases} proportionally to $\nu$
\item[$\Rightarrow$]<only@3> more points fall \textbf{inside} the $\varepsilon$-tube

\item[$\Rightarrow$]<only@3-> \notesonly{The fraction of points \textbf{outside} decreases as $\varepsilon$ increases, i.e} 
$\varepsilon \uparrow \;\;\leadsto\;\;$ \textcolor{magenta}{sum of errors} $\downarrow$

\only<3>{ 
\question{Does $\varepsilon$ grow indefinitely?}\\
}

\only<4-6>{\notesonly{{- No, }}$\varepsilon$ stops growing as soon as the \textcolor{magenta}{sum of errors} stops decreasing. But when?\\

\slidesonly{\underline{Hint}:}\notesonly{Remember from \eqref{eq:srmnu},} $\nu$-SVR \emph{optimizes} $\varepsilon$.} 

\only<5,6>{
	\begingroup
	\slidesonly{\scriptsize}
    \begin{align}
\only<5,6>{   
\frac{\partial}{\partial \varepsilon}
 \Big\lbrack 
 {\color{blue}\nu \varepsilon}& 
 + \frac{1}{p} {\color{magenta}\sum_{\alpha=1}^p (\varphi_\alpha + \varphi_\alpha^*)} 
 \Big\rbrack
 \eqexcl 0\notesonly{\\}\slidesonly{\quad}
 \Leftrightarrow \quad \frac{\partial}{\partial \varepsilon} {\color{blue}\nu \varepsilon} \notesonly{&}= -\frac{\partial}{\partial \varepsilon} 
 \Big(
 \frac{1}{p} {\color{magenta}\sum_{\alpha=1}^p (\varphi_\alpha + \varphi_\alpha^*)}
 \Big)
}
\only<6>{
 \intertext{At optimum, we have:}
 \Rightarrow \;  &-\frac{\partial}{\partial \varepsilon} 
 \Big(
 \overbrace{
 \frac{1}{p} {\color{magenta}\sum_{\alpha=1}^p (\varphi_\alpha + \varphi_\alpha^*)}
 }^{
 \text{fraction of errors}}
 \Big) \bigg|_{\varepsilon^{\mathrm{opt}}} = \nu{}
 }
\end{align}
\endgroup

\only<6>{
$\varepsilon$ grows as long as $\frac{1}{p} {\color{magenta}\sum_{\alpha=1}^p (\varphi_\alpha + \varphi_\alpha^*)} \le \nu$\\
$\nu$ is an \emph{upper bound} on the \emph{fraction of errors}.
}
}

\end{itemize}
\item<only@7-> As $\varepsilon$ \textbf{decreases}, starting from some large positive value:
\begin{itemize}
\item[$\Rightarrow$]<only@2-> \notesonly{The term }${\color{blue}\nu\varepsilon}$ \textbf{decreases} proportionally to $\nu$

\only<8->{
$\varepsilon$ decreases as long as $\frac{\#\mathrm{SV}}{p} \ge \nu$

\question{Where does $\frac{\#\mathrm{SV}}{p} \ge \nu$ come from?} 
}

\mode<article>{
%TODO FIX from here, this should lead to lower bound NOT upper bound
- Justifying this requires looking at the solution of the primal problem:
 
\begin{eqnarray}
				\frac{\partial L}{\partial \varepsilon} &=& \nu C - \smallsum{\alpha=1}{p} \lambda_\alpha
				- \smallsum{\alpha=1}{p} \lambda_\alpha^* - \delta
				\;\;\stackrel{!}{=}\;\; 0\\
				&\Rightarrow& \; \delta \;\;=\;\; \nu C - \smallsum{\alpha=1}{p} 
				(\lambda_\alpha + \lambda_\alpha^*) \\
				&\stackrel{\delta \geq 0}{\Rightarrow}&
				\nu C \;\geq\; \smallsum{\alpha=1}{p} 
				(\lambda_\alpha + \lambda_\alpha^*)
\end{eqnarray}
}

\only<9->{
\slidesonly{
 \begingroup
 \footnotesize
$$
				\frac{\partial L}{\partial \varepsilon} \stackrel{!}{=} 0\;\; \stackrel{\text{at } \varepsilon^{\mathrm{opt}}}{\Rightarrow} \;\; 
				\nu C \geq~{\color{red}\smallsum{\alpha=1}{p} 
				\overbrace{(\lambda_\alpha + \lambda_\alpha^*)}^{> 0 \text{ for SVs only}}} {\propto} \frac{\#\mathrm{SV}}{p}
				$$
\endgroup
}
}%only
\item[$\Rightarrow$]<only@10> $\nu$ acts as a \emph{lower bound} on the fraction of SVs, i.e. $\nu \le \frac{\# SV}{p}$.
\end{itemize}
\end{enumerate}


\end{frame}
