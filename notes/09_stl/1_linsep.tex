\section{Linear separability}


\begin{frame}\frametitle{\secname}

\underline{Setting}:\\

\begin{equation*}
\Big\{ \left(\vec x^{(\alpha)}, y^{(\alpha)}_{\mathrm{T}} \right) \Big\}\,,\quad \alpha = 1,\ldots,p
\end{equation*}

\begin{itemize}
\item $p$ points
\item $N$ dimensions $\vec x \in \R^N$
\item binary assignment: $y_T \in \{-1,1\}$
\end{itemize}

\pause

\underline{Model}:\\

Linear neuron with $N$ weights and a bias ($N+1$ degrees of freedom).

\pause

\underline{Task}:\\

Find the total number of binary assignments that are linearly separable.

\end{frame}

\begin{frame}\frametitle{Find the total number of affinely separable assignments}

\begin{enumerate}
\item Place $p$ points in $N$ dimensions, anyway you like as long as they're not colinear.
\item Assign all possible labels to the points.
\end{enumerate}

Example with 4 points in 2D:

\mode<presentation>{
\textbf{see blackboard}
}

\question{How many assignment configurations are there?}

\question{How many of them can be separated by a connectionist neuron?}

\end{frame}

\begin{frame}\frametitle{Calculate the number of configurations that are linearly separable}

Calculate the number of linearly separable assignments:

\begin{equation}
	\tilde C_{(p,N)} := 2 \sum_{k=0}^{N} 
	\underbrace{
	\left( \begin{array}{c}
	p-1\\
	k
	\end{array}\right)
	}_{\substack{\text{Binomial}\\ \text{coefficient}}}
\end{equation}

The Binomial coefficient\footnote{
See \href{https://en.wikipedia.org/wiki/Pascal\%27s_triangle}{Pascal's triangle} for a fun way on how to manually compute the coefficient.
} 
(``$n$ choose $k$'') 
is defined as:

\begin{equation}
\left(n \atop k \right) = \frac{n!}{k! (n-k)!}
\end{equation}

for $n,k \in \N_0$ with $n \ge k \ge 0$

\end{frame}

\begin{frame}\frametitle{The Binomial coefficient}

The Binomial coefficient represents the coefficient associated with the $y^k$ term in the polynomial expansion $(1+y)^n$.

Consider the following example\footnote{
The example is taken from the wikipedia article on \href{https://en.wikipedia.org/wiki/Binomial_coefficient}{Binomial coefficient}
} of using polynomial expansion for calculating the fourth power of $(1+y)$:

\begin{align}
(1+y)^4 &= 
\left({4 \atop 0}\right) y^0 +
\left({4 \atop 1}\right) y^1 +
{\color{blue}\left({4 \atop 2}\right) y^2} +
\left({4 \atop 3}\right) y^3 +
\left({4 \atop 4}\right) y^4\\
&= 1 + 4y + {\color{blue}6y^2} + 4y^3 + y^4
\label{eq:polyexpansion}
\end{align}

The binomial coefficient $\left({4 \atop 2}\right) = \frac{4!}{2!(4-2)!} = 6$. ${\color{blue}6}$ matches the coefficient in front of the ${\color{blue}y^2}$ term in the expansion\notesonly{ in \eqref{eq:polyexpansion}}.

\end{frame}

\begin{frame}\frametitle{How do we actually get the number of configurations?}

\begin{equation*}
	\tilde C_{(p,N)} := 2 \sum_{k=0}^{N} \left( \begin{array}{c}
	p-1\\
	k
	\end{array}\right)
\end{equation*}

Using our example with $p=4$ and $N=2$:
\begin{align}
	\tilde C_{(4,2)} &= 2 \sum_{k=0}^{2} \left( \begin{array}{c}
	4-1\\
	k
	\end{array}\right)\\
	&= 2 
	\Big\lbrack
	\underbrace{
	\left({4-1 \atop {\color{red}0}}\right)
	}_{\substack{\text{all positive}\\ \text{{\color{red}none} negative}}}
	+
	\underbrace{
	\left({4-1 \atop {\color{red}1}}\right)
	}_{\substack{\text{{\color{red}1} out of p}\\ \text{negative}}}
	+
	\underbrace{
	\left({4-1 \atop {\color{red}2}}\right)
	}_{\substack{\text{{\color{red}2} out of p}\\ \text{negative}}}
	\Big\rbrack\\
	&= 2
\end{align}


\end{frame}

\begin{frame}

\mode<presentation>{

\begin{equation*}
	\tilde C_{(p,N)} := 2 \sum_{k=0}^{N} \left( \begin{array}{c}
	p-1\\
	k
	\end{array}\right)
\end{equation*}
}

\textbf{But} this looks slightly different from what is in the lecture. In the lecture it looked like this:

\begin{equation}
	C_{(p,N)} := 2 \sum_{k=0}^{N-1} \left( \begin{array}{c}
	p-1\\
	k
	\end{array}\right)
\end{equation}

\question{Why does the sum for $\tilde C_{(p,N)}$ run to $N$ instead of $N-1$?}

\mode<article>{
-The reason that $\tilde C_{(p,N)}$ computes the sum for $k=1,\ldots,N$ instead of stopping at $N$ is that sum iterates over the degrees of freedom. 
Choosing a connectionist neuron to the separation involves $N+1$ degrees of freedom because we add the bias term (the ability to translate the hyperplane).
Therefore, the reason is the model choice and assuming that $C_{(p,N)}$ counts separability around the origin without the ability to shift the hyperplane.

-An alternative explanation to the difference in the formula is how to intepret $N$. The definition of $C_{(p,N)}$ treats $N$ as the parameters of the hyperplane. Therefore the bias is implicit.
If we assume that $N$ is the number of components in $\vec x$ and we need to prepend a bias term to it, we compute the number of separable configurations using $\tilde C_{(p,N)}$. 
It is important to understand that $\tilde C_{(p,N)}$ and $C_{(p,N)}$ effecitvley yield the same result when we feed each the appropriate arguments.
}
\end{frame}


