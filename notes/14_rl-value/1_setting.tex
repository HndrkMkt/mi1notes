\section{The Reinforcement Learning (RL) setting}


\definecolor{reward}{rgb}{0,0.5,0}
\definecolor{policy}{rgb}{0.75,0,0}
\definecolor{trans}{rgb}{0,0,1}

\begin{frame}\frametitle{\secname}

\only<1>{
Reminder:
\begin{itemize}
\item Data in supervised learning:
$
\Big\{
	\vec x^{(\alpha)} \in \R^N
	\,,\,
	\vec y_T^{(p)} \in \R^M
\Big\}_{\alpha=1}^{p}
$

\item Data in \textbf{un}supervised learning:
$
\Big\{
	\vec x^{(\alpha)} \in \R^N
\Big\}_{\alpha=1}^{p}
$\\

\end{itemize}
}

\only<1->{
Data in Reinforcement Learning:
}
\only<1>{
A sequence of random states $\vec x$ resulting from taking actions $\vec a$ over time which result in some reward $r$.
}
\pause
\only<2->{
\begin{itemize}
\item a state $\vec x \in \mathcal{X}$ or $\vec x \in \R^N$,\\
e.g. $\mathcal{X} := \{ \vec x_1, \ldots, \vec x_S\} \subset \{0,1\}^S$ (1-out-$S$ encoding)
\pause
\item an action $\vec a$ which can be taken by the agent:\\

$\vec a \in \mathcal{A}$ or $\vec a \in \R^M$,\\
 e.g. $\mathcal{A} := \{ \vec a_1, \ldots, \vec a_A\} \subset \{0,1\}^A$ (1-out-$A$ encoding)
\pause
\item a reward $r \in \R$ or $r \in \{0,1\}$,\\
 e.g. $r \in \{\text{``cheese''},\text{``no cheese''}\}$.

\end{itemize}
\pause
\mode<article>{Each time step describes what reward was received when performing some action while in some state. 
}
The sequence we observe becomes:

\slidesonly{\vspace{-4mm}}

\begin{align}
\label{eq:chain}
\left\{\vec x^{(t)}, \vec a^{(t)}, r^{(t)}\right\}_{t=0}^{p} = 
\left( \vec x^{(0)}, \vec a^{(0)}, r^{(0)} \right) \,,\, \ldots \,,\, \left(\vec x^{(p)}, \vec a^{(p)}, r^{(p)} \right)
\end{align}
}


\end{frame}

\begin{frame}

\only<1>{    
\begin{block}{Markov property}

The probability of transitioning form state $\vec x_i$ to $\vec x_j$ only depends on state $\vec x_i$ and the action $\vec a_k$ that was taken. It does not depend on earlier history.
\end{block}

This leads to the following }\textcolor{trans}{transition model (transition matrix)}:

\slidesonly{\vspace{-3mm}}

\begin{equation}
{
\color{trans}
P(\vec x_j | \vec x_i, \vec a_k)
}
\end{equation}
\only<1>{
which
\begin{itemize}
	\item measures the probability to end up in $\vec x_j$ 
		after choosing $\vec a_k$ when in $\vec x_i$,
	\item is a stationary distribution
	\begin{equation}
	P(\vec x_j^{(t+1+\tau)} | \vec x_i^{(t+\tau)}, \vec a_k^{(t+\tau)}) = P(\vec x_j^{(t+1)} | \vec x_i^{(t)}, \vec a_k^{(t)})\,,\quad \tau \in \{0,1,\ldots\}
	\end{equation}
\end{itemize}

and where
\begin{align}
{\color{trans}P(\vec x_j | \vec x_i, \vec a_k)} &\ge 0 \quad \forall i,j\\
\text{and}~ \sum_j {\color{trans}P(\vec x_j | \vec x_i, \vec a_k)} &= 1 \quad \forall i \quad \text{i.e. every row sums to 1}
\end{align}
}

\slidesonly{\vspace{-3mm}}

\question{What would $\sum_j {P(\vec x_j | \vec x_i, \vec a_k)} = 0 \;\; \forall \; k$ imply?}

\pause

- $\vec x_i$ \notesonly{is a state which leads nowhere, regardless of the action. It }is a \emph{terminal state}.

\question{What does an entry ${P(\vec x_j | \vec x_i, \vec a_k)} = 0$ represent?}

\pause

- When at $\vec x_i$, taking action $\vec a_k$ can never lead to $\vec x_j$.


\question{What does an entry ${P(\vec x_j | \vec x_i, \vec a_k)} = 0 \;\; \forall \; k$ represent?}

\pause

- $\vec a_k$ can never lead to $\vec x_j$, regardless of the action.

\question{What does an entry ${P(\vec x_j | \vec x_i, \vec a_k)} = 1$ represent?}

\pause

- When at $\vec x_i$, taking action $\vec a_k$ always leads to $\vec x_j$.

\end{frame}

\begin{frame}

\mode<presentation>{
\textcolor{trans}{transition model (transition matrix)}:

\begin{equation}
{
\color{trans}
P(\vec x_j | \vec x_i, \vec a_k)
}
\end{equation}
}

\begin{block}{Markov chain}
a sequence of random states $\vec x$ with a \emph{Markov property}.
\end{block}


\end{frame}

\begin{frame}

\begin{block}{\textcolor{policy}{Policy}}
Representation of the agent\footnote{
Without a policy we would only have a Markov Reward Process.
Having a policy (wanting to find a policy) turns the Markov Reward Process into a Markov Decision Process.
}. It measures the distribution over actions given states.
\begin{itemize}
\item map states to actions, i.e. the probability of doing $\vec a_k$ when in $\vec x_i$:

\begin{equation}
\color{policy}
\text{policy }\pi := \pi(\vec a_k | \vec x_i)
\end{equation}
\item the mapping can be stochastic or deterministic
\item the policy is stationary. That the policy changes during the training process does not contradict this.
It is regarded as a stationary because once trained and deployed, the way the agent decides does not change over time.
\end{itemize}

\end{block}

\end{frame}
