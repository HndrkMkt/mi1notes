\section{Gradient methods}

\subsection{The need for gradient-based optimization}

\begin{frame}


\notesonly{
We have some cost function we want to minimize but solving the problem analytically is not applicable.

}



\question{In what case(s) do we avoid an analytical solution?}

\begin{itemize}
	\item We cannot fit all the data into memory.
	\item It involves finding an inverse of a matrix, that is not invertible (e.g. linear regression $(\vec X \, \vec X^\top)^{-1}\vec X \, \vec y_{True}^\top$.
	\item There is no closed-form solution for the model we're using (e.g. MLP)
	\item Computing the Hessian is too costly.
	\item Non-stationary data. We cannot easily adapt to changes over time.
\end{itemize}

\end{frame}

We therefore opt for Gradient-based optimization which is an iterative optimization procedure. 
In the case of a \emph{minimization} problem one can reduce some cost function $E^T_{[\vec w]}$ by taking steps in the direction of steepest \emph{descent}.
For minimizing some cost function $E^T_{[\vec w]}$ w.r.t. to a set of weights $\vec w$, it follows that, if

\begin{equation}
\vec w_{t+1} = \vec w_t - \eta_t \, \frac{\partial E^T}{\partial \vec w} \Bigg|_{\vec w_t}
\end{equation}

and a small enough $\eta_t \in \R^+$, then the cost at the next step $t+1$ will become less or remain unchanged. That is $E^T_{[\vec w_{t+1}]} \le E^T_{[\vec w_{t}]}$.
We refer to $\eta_t$ as the learning rate. It allows us to control the size of the step we take in the direction of the gradient.
\emph{Subtracting} the gradient from the current $\vec w_t$ lets us move against the slope in towards some minimum. 

\subsection{``Scope'' of Gradient methods}.

The scope of the gradient method refers to the amount of data used for deciding on the direction of the gradient.



