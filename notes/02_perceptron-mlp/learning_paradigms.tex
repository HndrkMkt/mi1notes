\section{Learning Paradigms}

\begin{frame}

There are three learning paradigms in machine learning:
\begin{itemize}
\item Supervised learning\notesonly{. This is learning with a teacher (cf. \sectionref{sec:supervised}).}
\item Unsupervised learning\notesonly{. Learning only from observations, without a teacher (cf. \sectionref{sec:unsupervised}).}
\item Reinforcement learning\notesonly{. This is about maximizing reward (cf. \sectionref{sec:reinforcement}).}
\end{itemize}
\mode<article>{
It is possible to understand the differences between them by describing
\begin{inparaenum}[(i)]
    \item the data each of them deals with and
    \item the objective that each is trying to model
\end{inparaenum}
}
\mode<presentation>{

Differences: Data \& objective model.
}

\end{frame}


\subsection{Supervised learning} \label{sec:supervised}

\begin{frame}\frametitle{\subsecname}

\mode<article>Supervised learning is\mode<all>essentially function fitting.

\underline{Data}:

Consider this dataset of tuples:
\begin{equation}
\label{eq:tuples}
\left( \vec x^{(1)}, \vec y_T^{(1)} \right) \,,\, \ldots \,,\, \left( \vec x^{(\alpha)}, \vec y_T^{(\alpha)} \right) \,,\, \ldots \,,\, \left( \vec x^{(p)}, \vec y_T^{(p)} \right)
\end{equation}

\mode<article>{
where,
\begin{itemize}
\item[] $\vec x$ denotes the observation with $N$ components. $\vec x = (x_1, x_2, \ldots x_N)$, $\vec x \in \R^N$ (e.g. pixel values in an image)
\item[] $\vec y_T$ denotes the label assigned to this observation. It is also referred to as the ground truth.
When $y_T \in \R$ or $\vec y_T \in \R^M$, we are looking at a regression problem (e.g. predicting house prices, object localization - bounding box regression (x, y, w, h)).\\
When $y_T \in \{ 0, 1\}$, we refer to this as a binary classification problem (e.g. is this a cat or a dog).\\
When $y_T \in \{ 0, \ldots, K-1\}$, we refer to this as a multi-class classification problem (e.g. which letter is this?).
\item[] $p$ denotes the size of the dataset (i.e. how many labeled observations we have.)
\item[] The superscript $^{(\alpha)}$ denotes the index of a particular sample within the dataset. $\alpha = 1 , \ldots, p$
\item[] The samples are often assumed to be \emph{independent and identically distributed} (\iid).
\end{itemize}
}

\end{frame}

\begin{frame}

\underline{Objective model}:

Supervised learning algorithms are used to find a function that maps observations to their label which represents some concept or value.
This mapping can be described in the form of:
\begin{itemize}
\item a discrimination function $\vec y = \vec f(\vec x)$, \\ 

where $\vec f(\cdot)$ denotes a vector valued function $\vec f: \R^N \rightarrow \R^M$
\item a conditional distribution $P(\,\vec y \,|\, \vec x\,)$.
\end{itemize}

\end{frame}

\subsection{Unsupervised learning} \label{sec:unsupervised}

\begin{frame}\frametitle{\subsecname}

\mode<article>
Unsupervised learning tries to find 
\mode<all>
 interesting directions and/or structure in the data using only observations $\vec x \in \R^N$.

\underline{Data}:

A dataset of observations:
\begin{equation}
\label{eq:observations}
%\vec x^{(1)} \,,\, \ldots \,,\, \vec x^{(\alpha)} \,,\, \ldots \,,\, \vec x^{(p)}
\vec X = 
\left(
\begin{array}{cccccc}
\Big| & \Big| & & \Big| & & \Big| \\[3mm]
\vec x^{(1)} & \vec x^{(2)} & \cdots & \vec x^{(\alpha)} & \cdots & \vec x^{(p)}\\[2mm]
\Big| & \Big| & & \Big| & & \Big|
\end{array}
\right) \in \R^{N \times p}
\end{equation}

where $p$ denotes the number of observations (i.e. size of the dataset) and $N$ denotes the number of dimensions.
The samples are often assumed to be \iid.

Example: $\vec x$ could represent user ratings, pixel values in images.

\end{frame}
\begin{frame}

\underline{Objective model}:

An unsuperivsed learning algorithms are used to capture structure or directions in the data. This can be achieved by finding:
\begin{itemize}
\item the underlying distribution $P(\vec x)$ that generated this data (e.g. density estimation),
\item $\vec z := \vec f(\vec x)$, where $\vec z$ is a measure of 
\begin{itemize}
\item possible structure such as clustering or grouping in the data ($\vec z \in {0,\ldots,K-1}$), \\

and/or

\item possible directions in the data, by finding another continuous space for descrbing this data. \\

Example: dimensionalty reduction, $\vec z \in \R^M$ with $M < N$.

\end{itemize}
\end{itemize}

\end{frame}

\newpage

\subsection{Reinforcement learning} \label{sec:reinforcement}

\begin{frame}\frametitle{\subsecname}


Reinforcement learning is a about what actions to take in which situations in order to maximize some reward.
This implies that learning happens through an interaction of the agent with its environment.

\end{frame}

\begin{frame}

\underline{Data}:

Our data consists of a sequence. Each time step is described by 

\begin{itemize}
\item a state $\vec x \in \mathcal{X}$ or $\vec x \in \R^N$, e.g. $\mathcal{X} := \{ \vec x_1, \ldots, \vec x_S\} \subset \{0,1\}^S$ (1-out-$S$ encoding)
\item an action $\vec a$ which can be taken by the agent:\\

$\vec a \in \mathcal{A}$ or $\vec a \in \R^M$, e.g. $\mathcal{A} := \{ \vec a_1, \ldots, \vec a_A\} \subset \{0,1\}^A$ (1-out-$A$ encoding)

\item a reward $r \in \R$ or $r \in \{0,1\}$, e.g. $r \in \{\text{``cheese''},\text{``no cheese''}\}$.

\end{itemize}

Each time step describes what reward was received when performing some action while in some state. 
The sequence we observe becomes:

\begin{align}
\label{eq:chain}
\left\{\vec x^{(t)}, \vec a^{(t)}, r^{(t)}\right\}_{t=0}^{p} = 
\left( \vec x^{(0)}, \vec a^{(0)}, r^{(0)} \right) \,,\, \ldots \,,\, \left(\vec x^{(p)}, \vec a^{(p)}, r^{(p)} \right)
\end{align}




Since we observe sequential data, we no longer assume \iid \emph{within} a sequence. But if we have multiple sequences, we can assume \iid \emph{across} sequences.

\end{frame}

\begin{frame}

Example:
\begin{itemize}
\item $\vec x \; \corresponds$ position in a maze
\item $\vec s \; \corresponds$ movement direction (velocity), e.g. turn left/right
\item $r \; \corresponds$ found the cheese, found smaller piece of cheese. A negative reward, ``punishment'' is also possible.
\end{itemize}
\end{frame}

\begin{frame}

\underline{Objective model}:

Reinforcement learning can be used to:
\begin{itemize}
\item select the optimal action when arriving at a state: $\vec a^* = \vec f(\vec x, r)$
\item find the transition model $P(\vec x^{(t)}\,|\,\vec x{(t-1)}, \vec a^{(t-1)})$, where $t$ is a time step within the sequence.
\end{itemize}
\pause
\question{Would you regard reinforcement learning as supervised or unsupervised learning?}

\end{frame}

\newpage

\subsection{Summarized overview}

\begin{frame}

\begin{table}[!h]

\makebox[1 \textwidth][c]{       %centering table
\resizebox{0.85 \textwidth}{!}{   %resize table
\label{tab:summary} 
\begin{tabular}{l||l|l}
                       & \multicolumn{1}{c|}{Data}                                                   & \multicolumn{1}{c}{Objective model}                                                                                                 \\ \hline \hline
\begin{tabular}[c]{@{}l@{}}Supervised\\ learning\end{tabular}    & \begin{tabular}[c]{@{}l@{}}\multicolumn{1}{c}{$\left( \vec x^{(1)}, \vec y_T^{(1)} \right) \,,\, \ldots  \,,\, \left( \vec x^{(p)}, \vec y_T^{(p)} \right)$}
\\[2mm] \qquad\qquad \rotatebox[origin=c]{180}{$\Lsh$} label $\R$, $\R^M$,$\{0,1\}$,\\ \qquad\qquad\qquad$\{0,\ldots,K-1\}$\\[2mm] \,\quad\, \rotatebox[origin=c]{180}{$\Lsh$} observation $\R^N$, often iid.\end{tabular} & \begin{tabular}[c]{@{}l@{}}\\[1mm]discrimination function\\ $\vec y = \vec f(\vec x)$\\ \\ conditional distribution\\ $P(\vec y | \vec x)$\\[1mm]\end{tabular}                                                   \\ \hline
\begin{tabular}[c]{@{}l@{}}Unsupervised\\ learning\end{tabular}   & \begin{tabular}[c]{@{}l@{}}\multicolumn{1}{c}{$\vec x^{(1)},\ldots,\vec x^{(p)} \in \R^N$}\\[2mm] observations often iid.\end{tabular}         & \begin{tabular}[c]{@{}l@{}}\\generative model/\\ data distribution $P(\vec x)$ \\[2mm] clustering: \\ $f(\vec x): \R^N \mapsto \{0,\ldots,K-1\}$\\[2mm] dimensionality reduction: \\ $\vec f(\vec x): \R^N \mapsto \R^M $\\[2mm]\end{tabular} \\ \hline
\begin{tabular}[c]{@{}l@{}}Reinforcement\\ learning\end{tabular}  & \begin{tabular}[c]{@{}l@{}} \\ \multicolumn{1}{c}{$\left( \vec x^{(0)}, \vec a^{(0)}, r^{(0)} \right) \,,\, \ldots \,,\, \left(\vec x^{(p)}, \vec a^{(p)}, r^{(p)} \right)$}\\[2mm] \;\qquad\qquad\quad \rotatebox[origin=c]{180}{$\Lsh$} reward $r \in \R$\\ \;\qquad\quad \rotatebox[origin=c]{180}{$\Lsh$} action $\vec a \in \R^M$ or $\{0,1\}^A$\\ \;\quad \rotatebox[origin=c]{180}{$\Lsh$} state $\vec x \in \R^N$ or $\{0,1\}^S$\\[2mm] \end{tabular}   & \begin{tabular}[c]{@{}l@{}}select optimal action:\\ $\vec a^* = \vec f(\vec x,r)$\\[1mm] transition model:\\ $P(\vec x^{(t)}\,|\,\vec x{(t-1)}, \vec a^{(t-1)})$\end{tabular}                                                          
\\ \hline
\end{tabular}
} %close resize
} %close centering
%\caption{Summary of learning paradigms} % makes table shift in a strange way
\end{table}

\end{frame}


