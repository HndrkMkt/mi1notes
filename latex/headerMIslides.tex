\documentclass[pdfetex,ignorenonframetext]{beamer}
\usetheme{TUBerlin}
\usepackage{latexsym,listings,graphicx}
\usepackage[latin1]{inputenc}
\usepackage[T1]{fontenc}
%\usepackage[pdfetex]{preview}
\usepackage{hyperref}
\usepackage[overlay,absolute]{textpos}
\usepackage{algorithmic}
\usepackage{verbatim}
\usepackage{amssymb,amsmath}
\usepackage{enumerate}
\usepackage{mathtools}
\usepackage[ruled]{algorithm2e}
\usepackage{appendixnumberbeamer}
\usepackage[mathscr]{euscript}
\usepackage{caption}
\usepackage{subcaption}
\newsavebox{\imagebox}
\hypersetup{colorlinks=true,linkcolor=black,urlcolor=blue,citecolor=green}
%\usepackage{paralist}
\usepackage{xcolor,cancel}

\usepackage{tikz}
\usetikzlibrary{shapes.geometric}
\tikzset{myrect/.style={rectangle, fill=#1!20, draw=#1!75, text=black}}
\tikzstyle{axes}=[]

% from tex.stackexchange useful to align substacks

\makeatletter
\newcommand{\subalign}[1]{%
  \vcenter{%
    \Let@ \restore@math@cr \default@tag
    \baselineskip\fontdimen10 \scriptfont\tw@
    \advance\baselineskip\fontdimen12 \scriptfont\tw@
    \lineskip\thr@@\fontdimen8 \scriptfont\thr@@
    \lineskiplimit\lineskip
    \ialign{\hfil$\m@th\scriptstyle##$&$\m@th\scriptstyle{}##$\crcr
      #1\crcr
    }%
  }
}
\makeatother

\newcommand{\misubsection}[1]{
	\subsection{\thesubsection) #1}
	\begin{frame} 
		\begin{center} \huge
			\misection.\thesubsection~#1
		\end{center}
	\end{frame}
}

\newenvironment<>{varblock}[2][.9\textwidth]{%
  \setlength{\textwidth}{#1}
  \begin{actionenv}#3%
    \def\insertblocktitle{#2}%
    \par%
    \usebeamertemplate{block begin}}
  {\par%
    \usebeamertemplate{block end}%
  \end{actionenv}}

\setbeamertemplate{caption}{\raggedright\insertcaption\par}

\setbeamertemplate{section in toc}[default]
\setbeamertemplate{subsection in toc}[square]
\setbeamertemplate{enumerate items}[circle] 
\setbeamertemplate{itemize items}[square] 
\setbeamertemplate{navigation symbols}{}
\usefonttheme[onlymath]{serif}

\title[MI 1 Tutorial]{Machine Intelligence 1 Tutorial}
\institute[www.ni.tu-berlin.de]{Fachgebiet Neuronale Informationsverarbeitung (NI)}
\author[Kashef]{Youssef Kashef}
\date{WS 19/20}

\graphicspath{{../../../../../figures/pdf/}}

\newcommand{\figref}[1]{Fig. \hspace{-0.5mm}\ref{#1}}
\newcommand{\figureref}[1]{Figure \hspace{-0.5mm}\ref{#1}}

\newcommand{\sectionref}[1]{Section \hspace{-1.0mm}\ref{#1}}

% TODO consolidate with notes
\newcounter{sheetno}

\newcounter{question}
\setcounter{question}{1}

\newcommand{\question}[2]{%
	\vspace{4mm}%
	{Q\thesheetno.\arabic{question}: ~ \textit{#1}%
	}\\[5mm]%
	\addtocounter{question}{1}%
}

% Include in notes but exclude from slides
% see corresponding \slidessonly for opposite behavior
\newcommand{\notesonly}[1]{}

\newcommand\labelitemi{\textbullet}

\newcommand{\inparaenum}{\enumerate}
